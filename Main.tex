% Options for packages loaded elsewhere
\PassOptionsToPackage{unicode}{hyperref}
\PassOptionsToPackage{hyphens}{url}
\PassOptionsToPackage{dvipsnames,svgnames,x11names}{xcolor}
%
\documentclass[
  letterpaper,
  DIV=11,
  numbers=noendperiod]{scrartcl}

\usepackage{amsmath,amssymb}
\usepackage{iftex}
\ifPDFTeX
  \usepackage[T1]{fontenc}
  \usepackage[utf8]{inputenc}
  \usepackage{textcomp} % provide euro and other symbols
\else % if luatex or xetex
  \usepackage{unicode-math}
  \defaultfontfeatures{Scale=MatchLowercase}
  \defaultfontfeatures[\rmfamily]{Ligatures=TeX,Scale=1}
\fi
\usepackage{lmodern}
\ifPDFTeX\else  
    % xetex/luatex font selection
  \setmainfont[]{Inter}
  \setsansfont[]{Inter}
  \setmathfont[]{Fira Math}
\fi
% Use upquote if available, for straight quotes in verbatim environments
\IfFileExists{upquote.sty}{\usepackage{upquote}}{}
\IfFileExists{microtype.sty}{% use microtype if available
  \usepackage[]{microtype}
  \UseMicrotypeSet[protrusion]{basicmath} % disable protrusion for tt fonts
}{}
\makeatletter
\@ifundefined{KOMAClassName}{% if non-KOMA class
  \IfFileExists{parskip.sty}{%
    \usepackage{parskip}
  }{% else
    \setlength{\parindent}{0pt}
    \setlength{\parskip}{6pt plus 2pt minus 1pt}}
}{% if KOMA class
  \KOMAoptions{parskip=half}}
\makeatother
\usepackage{xcolor}
\setlength{\emergencystretch}{3em} % prevent overfull lines
\setcounter{secnumdepth}{5}
% Make \paragraph and \subparagraph free-standing
\ifx\paragraph\undefined\else
  \let\oldparagraph\paragraph
  \renewcommand{\paragraph}[1]{\oldparagraph{#1}\mbox{}}
\fi
\ifx\subparagraph\undefined\else
  \let\oldsubparagraph\subparagraph
  \renewcommand{\subparagraph}[1]{\oldsubparagraph{#1}\mbox{}}
\fi


\providecommand{\tightlist}{%
  \setlength{\itemsep}{0pt}\setlength{\parskip}{0pt}}\usepackage{longtable,booktabs,array}
\usepackage{calc} % for calculating minipage widths
% Correct order of tables after \paragraph or \subparagraph
\usepackage{etoolbox}
\makeatletter
\patchcmd\longtable{\par}{\if@noskipsec\mbox{}\fi\par}{}{}
\makeatother
% Allow footnotes in longtable head/foot
\IfFileExists{footnotehyper.sty}{\usepackage{footnotehyper}}{\usepackage{footnote}}
\makesavenoteenv{longtable}
\usepackage{graphicx}
\makeatletter
\def\maxwidth{\ifdim\Gin@nat@width>\linewidth\linewidth\else\Gin@nat@width\fi}
\def\maxheight{\ifdim\Gin@nat@height>\textheight\textheight\else\Gin@nat@height\fi}
\makeatother
% Scale images if necessary, so that they will not overflow the page
% margins by default, and it is still possible to overwrite the defaults
% using explicit options in \includegraphics[width, height, ...]{}
\setkeys{Gin}{width=\maxwidth,height=\maxheight,keepaspectratio}
% Set default figure placement to htbp
\makeatletter
\def\fps@figure{htbp}
\makeatother

\usepackage{amsmath, xparse}
\usepackage{fancyvrb, fvextra}
\usepackage{amssymb}
\usepackage{graphicx}
\usepackage{bm}
\usepackage{svg}
\usepackage{listings}
\usepackage{tikz}
\usepackage{multicol}
\usepackage{xifthen}
\DefineVerbatimEnvironment{Highlighting}{Verbatim}{breaklines,commandchars=\\\{\}}
\KOMAoption{captions}{tableheading}
\makeatletter
\makeatother
\makeatletter
\makeatother
\makeatletter
\@ifpackageloaded{caption}{}{\usepackage{caption}}
\AtBeginDocument{%
\ifdefined\contentsname
  \renewcommand*\contentsname{Table of contents}
\else
  \newcommand\contentsname{Table of contents}
\fi
\ifdefined\listfigurename
  \renewcommand*\listfigurename{List of Figures}
\else
  \newcommand\listfigurename{List of Figures}
\fi
\ifdefined\listtablename
  \renewcommand*\listtablename{List of Tables}
\else
  \newcommand\listtablename{List of Tables}
\fi
\ifdefined\figurename
  \renewcommand*\figurename{Figure}
\else
  \newcommand\figurename{Figure}
\fi
\ifdefined\tablename
  \renewcommand*\tablename{Table}
\else
  \newcommand\tablename{Table}
\fi
}
\@ifpackageloaded{float}{}{\usepackage{float}}
\floatstyle{ruled}
\@ifundefined{c@chapter}{\newfloat{codelisting}{h}{lop}}{\newfloat{codelisting}{h}{lop}[chapter]}
\floatname{codelisting}{Listing}
\newcommand*\listoflistings{\listof{codelisting}{List of Listings}}
\makeatother
\makeatletter
\@ifpackageloaded{caption}{}{\usepackage{caption}}
\@ifpackageloaded{subcaption}{}{\usepackage{subcaption}}
\makeatother
\makeatletter
\@ifpackageloaded{tcolorbox}{}{\usepackage[skins,breakable]{tcolorbox}}
\makeatother
\makeatletter
\@ifundefined{shadecolor}{\definecolor{shadecolor}{rgb}{.97, .97, .97}}
\makeatother
\makeatletter
\makeatother
\makeatletter
\makeatother
\ifLuaTeX
  \usepackage{selnolig}  % disable illegal ligatures
\fi
\IfFileExists{bookmark.sty}{\usepackage{bookmark}}{\usepackage{hyperref}}
\IfFileExists{xurl.sty}{\usepackage{xurl}}{} % add URL line breaks if available
\urlstyle{same} % disable monospaced font for URLs
\hypersetup{
  pdftitle={Why CyberDragons?},
  pdfauthor={GSMST CyberDragons; GSMST Computer Science Club},
  colorlinks=true,
  linkcolor={blue},
  filecolor={Maroon},
  citecolor={Blue},
  urlcolor={Blue},
  pdfcreator={LaTeX via pandoc}}

\title{Why CyberDragons?}
\author{GSMST CyberDragons \and GSMST Computer Science Club}
\date{}

\begin{document}
\maketitle
\begin{tikzpicture}[overlay, remember picture]
  \node[anchor=center, opacity=0.2] at ([yshift=-2cm]current page.center) {\includegraphics[width=2\paperwidth]{CyberDragons.png}};
\end{tikzpicture}
\ifdefined\Shaded\renewenvironment{Shaded}{\begin{tcolorbox}[interior hidden, enhanced, borderline west={3pt}{0pt}{shadecolor}, sharp corners, frame hidden, breakable, boxrule=0pt]}{\end{tcolorbox}}\fi

\renewcommand*\contentsname{Table of Contents}
{
\hypersetup{linkcolor=}
\setcounter{tocdepth}{4}
\tableofcontents
}
\newpage{}

\hypertarget{who-we-are}{%
\section{Who We Are}\label{who-we-are}}

\emph{CyberDragons} is a competitive cybersecurity organization from the
Gwinnett School of Math, Science, and Technology (GSMST) seeking to
nurture the next generation of ethical hacking professionals. Our
dedication and relentless pursuit of knowledge in this ever-evolving
field have set us apart from our competition, making us a force to be
reckoned with. We take pride in building our members from the ground up,
training them to excel at what they do.

\hypertarget{past-achievements}{%
\section{Past Achievements}\label{past-achievements}}

\begin{itemize}
\tightlist
\item
  7th nationally in picoCTF (2022)
\item
  2nd nationally in CyberPatriot's Silver division (2022-2023)
\item
  1st and 2nd in Georgia in CyberPatriot's Gold division (2022-2023)
\item
  3 national CyberPatriot semifinalists (2022-2023)
\end{itemize}

\hypertarget{competitions}{%
\section{Competitions}\label{competitions}}

\begin{itemize}
\tightlist
\item
  CyberPatriot
\item
  picoCTF
\item
  CSAW CTF
\item
  ASIS CTF
\item
  CyberStart America
\item
  DiceCTF
\item
  LA CTF
\item
  US Cyber Challenge: Cyber Quests
\item
  Lockheed Martin's CYBERQUEST
\item
  Technology Student Association Cybersecurity
\item
  Cyber Skyline's National Cyber League (NCL)
\item
  ÅngstromCTF
\item
  TJCTF
\end{itemize}

\hypertarget{what-we-teach}{%
\section{What We Teach}\label{what-we-teach}}

\hypertarget{cybersecurity-fundamentals}{%
\subsection{Cybersecurity
Fundamentals}\label{cybersecurity-fundamentals}}

\begin{itemize}
\tightlist
\item
  Operating system security and vulnerabilities
\item
  Cryptography and encryption techniques
\item
  Basic computer networks and protocols
\end{itemize}

\hypertarget{incident-response-and-forensics}{%
\subsection{Incident Response and
Forensics}\label{incident-response-and-forensics}}

\begin{itemize}
\tightlist
\item
  Detecting and responding to cybersecurity incidents
\item
  Digital forensics techniques and tools
\end{itemize}

\hypertarget{network-and-application-security}{%
\subsection{Network and Application
Security}\label{network-and-application-security}}

\begin{itemize}
\tightlist
\item
  Firewall configuration and management
\item
  Intrusion detection and prevention
\item
  Network traffic monitoring and analysis
\item
  Writing secure code to prevent common threat vectors
\end{itemize}

\hypertarget{ethical-hacking}{%
\subsection{Ethical Hacking}\label{ethical-hacking}}

\begin{itemize}
\tightlist
\item
  Hands-on experience with penetration testing tools such as Metasploit,
  Burp Suite, and Nmap
\item
  Exploiting common vulnerabilities such as SQL injection and XSS
\item
  Reverse engineering and binary exploitation techniques
\end{itemize}

\hypertarget{best-security-practices}{%
\section{Best Security Practices}\label{best-security-practices}}

\begin{itemize}
\tightlist
\item
  Principle of least privilege
\item
  Common attack vectors and how to prevent them
\item
  How to secure your personal devices and accounts
\item
  Online safety and privacy tips
\item
  Awareness on phishing techniques and social engineering
\end{itemize}

\newpage{}

\hypertarget{what-we-need}{%
\section{What We Need}\label{what-we-need}}

\hypertarget{financial-contributions}{%
\subsection{Financial Contributions}\label{financial-contributions}}

CyberDragons pride in its history of outstanding achievements in
cybersecurity competitions. However, many of our dedicated competitors
come from economically disadvantaged backgrounds, with many relying on
Free and Reduced meal plans. The financial constraints associated with
travel costs and entry fees limit our ability to send more than one team
for many competitions. This limitation puts us at a significant
disadvantage compared to schools with larger budgets.

\hypertarget{professional-coaching-and-mentorship}{%
\subsection{Professional Coaching and
Mentorship}\label{professional-coaching-and-mentorship}}

While our school lacks a dedicated cybersecurity program or courses, our
team continues to excel and distinguish itself even against schools with
such programs. However, we recognize that to reach our full potential,
our students need guidance from experts in the field. Although our
seminars and competitions provide valuable experience, having tutors and
mentors would give us a competitive edge over other teams.

\hypertarget{field-and-work-experience}{%
\subsection{Field and Work Experience}\label{field-and-work-experience}}

GSMST offers two experiential learning courses crucial for student
graduation: the \emph{Junior Fellowship Experience} and the \emph{Senior
Capstone Experience}. The Junior Fellowship Experience serves as an
introduction to workforce and professional development and spans one
semester. The Senior Capstone Experience represents the culmination of
our students' high school journey, involving year-long collaboration
with a partnering company.

We believe our members would be outstanding interns in your company,
especially in the realm of cybersecurity. Their passion and natural
aptitude for the field, combined with real-world experience gained in
your organization, would not only benefit them but also enrich your
company with fresh perspectives and innovative ideas.

\newpage{}

\hypertarget{how-you-can-contribute}{%
\section{How You Can Contribute}\label{how-you-can-contribute}}

\hypertarget{donations}{%
\subsection{Donations}\label{donations}}

By donating to CyberDragons, you would enable us to attend these
competitions in person, which provides invaluable networking
opportunities and enhanced learning experiences. Additionally, your
contributions would allow us to send multiple teams to a variety of
competitions, giving our talented students more opportunities to
showcase their skills and compete at the highest level. If you wish to
make a contribution, please make checks payable to \emph{GSMST Computer
Science Club} and send them to the following address:

GSMST Computer Science Club 970 McElvaney Ln NW, Lawrenceville, GA
30044.

For any further inquiries regarding payment, please contact our adult
sponsor \href{mailto:Julia.Rachkovskiy@gcpsk12.org}{Julia Rachkovskiy}.

\hypertarget{mentors}{%
\subsection{Mentors}\label{mentors}}

We are also seeking mentors who can provide training in various aspects
of cybersecurity, including ethical hacking, penetration testing, and
offensive security techniques. Your employees' expertise and guidance
could significantly enhance our team's skills and performance, ensuring
our continued success in competitions.

\hypertarget{internship-opportunities}{%
\subsection{Internship Opportunities}\label{internship-opportunities}}

Finally, we kindly request your company to consider providing internship
opportunities for our students. By taking them under your wing and
teaching them cybersecurity and networking skills, you would be
empowering the next generation of cybersecurity professionals while
gaining dedicated and talented individuals to contribute to your
organization's success.

\hypertarget{conclusion}{%
\subsection{Conclusion}\label{conclusion}}

Your support, whether financial, mentorship, or internship
opportunities, would greatly contribute to our mission of nurturing
ethical hacking professionals and ensuring their success in competitions
and beyond. Thank you for considering our partnership request in this
educational endeavor.

\emph{-- GSMST CyberDragons}



\end{document}
